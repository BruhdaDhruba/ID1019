\documentclass[a4paper,11pt]{article}

\usepackage[utf8]{inputenc}


\begin{document}

\title{
    \textbf{Derivatives}
}
\author{Dhruba Das}
\date{Spring Term 2024}

\maketitle

\section*{Introduction: Problems to solve}

There are in total 3 problems I had to solve in this assignment: Implementing the types of data, implementing the rules of the derivatives,  and simplification of the obtained expressions-

\section*{Data types}

I had to define the types for numbers, atoms and different expressions, as shown below:

\begin{verbatim}
@type literal() :: {:num, number()} 
  | {:var, atom()}
  
  @type expr() :: {:add, expr(), expr()}
  | {:mul, expr(), expr()}
  | {:exp, expr(), literal()}
  | {:ln, expr()}
  | {:sin, expr()}
  | {:cos, expr()}
  | literal()
\end{verbatim}

\section*{Rules of derivation}
This wasn't actually the trickiest part of the task, with only adding the four rules for derivation, along with new methods with minor changes which use recursion to derive/differentiate the given expression; note that the reciprocal and square roots derivation is done using the $:exp$ data type. Below is the code:

\begin{verbatim}
def deriv({:num, _}, _) do {:num, 0} end
  def deriv({:var, v}, v) do {:num, 1} end
  def deriv({:var, _}, _) do {:num, 0} end
  
  def deriv({:add, e1, e2}, v) do
    {:add, deriv(e1,v), deriv(e2,v)}
  end
  def deriv({:mul, e1, e2}, v) do
    {:add, 
     {:mul, deriv(e1, v), e2},
      {:mul, e1, deriv(e2, v)}}
  end
  def deriv({:exp, e, {:num, n}}, v) do 
    {:mul,
     deriv(e,v),
      {:mul, {:num, n}, {:exp, e, {:num, (n-1)}}}}
  end
  def deriv({:ln, e}, v) do
    {:mul,
     deriv(e, v), {:exp, e, {:num, -1}}}
  end
  def deriv({:sin, e}, v) do
    {:mul, deriv(e, v), {:cos, e}}
  end
\end{verbatim}


\section*{Simplification}
This was the toughest part of the task for me, although not actually tough since I'd gone through all 6 instruction videos before attempting the assignment. Shown below is the code for simplifying different expressions, including those for addition, multiplication, natural log ($ln$), and $sin$ and $cos$ (since there is a lot of code, I'll only be including the code for the crucial bits, since other lines of code are there for pattern-matching)

\begin{verbatim}
  def simplify({:num, n}) do {:num, n} end
  def simplify({:var, v}) do {:var, v} end
  ...
  def simplify({:add, e1, e2}) do
    simplify_add(simplify(e1), simplify(e2))
  end
  def simplify({:mul, e1, e2}) do
    simplify_mul(simplify(e1), simplify(e2))
  end
  def simplify({:exp, e1, e2}) do
    simplify_exp(simplify(e1), simplify(e2))
  end
  ...
  def simplify_mul({:num, 0}, _) do {:num, 0} end
  ...
  def simplify_mul({:num, 1}, e2) do e2 end
  ...
  def simplify_mul({:num, n1}, {:num, n2}) do {:num, n1*n2} end
  ...
  def simplify_mul({:num, n1}, {:mul, {:num, n2}, e2}) do 
    {:mul, {:num, n1*n2}, e2} 
  end
  ...
  def simplify_mul({:mul, {:num, n2}, e2}, {:num, n1}) do 
    {:mul, {:num, n1*n2}, e2} 
  end
  def simplify_mul({:mul, e1, {:num, n2}}, {:num, n1}) do 
    {:mul, {:num, n1*n2}, e1} 
  end

  def simplify_mul(e1, e2) do {:mul, e1, e2} end
  ...
  def simplify_exp(_, {:num, 0}) do {:num, 1} end
  def simplify_exp(e1, {:num, 1}) do e1 end
  def simplify_exp({:num, 0}, _) do {:num, 0} end
  def simplify_exp({:num, 1}, _) do {:num, 1} end
  def simplify_exp({:num, n1}, {:num, n2}) do {:num, :math.pow(n1,n2)} end
  def simplify_exp(e1, e2) do {:exp, e1, e2} end
\end{verbatim} 

\section*{Print functions}
Shown below are also the print functions which I implemented:
\begin{verbatim}
  def pprint({:num, n}) do "#{n}" end
  def pprint({:var, v}) do "#{v}" end
  def pprint({:add, e1, e2}) do "(#{pprint(e1)} + #{pprint(e2)})" end
  def pprint({:mul, e1, e2}) do "#{pprint(e1)} * #{pprint(e2)}" end
  def pprint({:exp, e1, e2}) do "#{pprint(e1)}^(#{pprint(e2)})" end
  def pprint({:ln, e1}) do "ln(#{pprint(e1)})" end
  def pprint({:sin, e1}) do "sin(#{pprint(e1)})" end
  def pprint({:cos, e1}) do "cos(#{pprint(e1)})" end
\end{verbatim} 



\section*{An example}
Shown below is an example which uses the rules and simplification implemented:
\begin{verbatim}
def test1() do
  b = {:mul, {:num, 1}, {:exp, {:sin, {:mul, {:var, :x}, {:num, 2}}}, {:num, -1}}}
  c = deriv(b, :x)
  d = simplify(c)
  IO.write("expression: #{pprint(b)} \n")
  IO.write("derivative: #{pprint(c)} \n")
  IO.write("simplified: #{pprint(d)} \n")
  d
 end
\end{verbatim} 

Which gives us the output:
\begin{verbatim}
expression: 1 * sin(x * 2)^(-1) 

derivative: 
(0 * sin(x * 2)^(-1) + 1 * (1 * 2 + x * 0) * cos(x * 2) * -1 * sin(x * 2)^(-2)) 

simplified: 2 * cos(x * 2) * -1 * sin(x * 2)^(-2) 
\end{verbatim} 





\end{document}
