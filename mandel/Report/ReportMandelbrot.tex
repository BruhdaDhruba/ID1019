\documentclass[a4paper,11pt]{article}

\usepackage[utf8]{inputenc}
\usepackage{subcaption}
\usepackage{booktabs}
\usepackage{caption}
\usepackage{amsmath}
\usepackage{tkz-euclide}
\usepackage{hyperref}
\usepackage{graphicx}


\begin{document}

\title{
    \textbf{Mandelbrot}
}
\author{Dhruba Das}
\date{Spring Term 2024}

\maketitle

\section*{Introduction}
In this assignment, we take a look at complex numbers, and the Mandelbrot set; it's defined as the set of complex numbers $c$ for which the sequence $z_n$ does not approach infinity; in the sequence, $z_0 = 0$ and $z_{n+1} = z_n^2 + c$.

Our main goal is to use this particular set of complex numbers to generate an image.


\section*{Cmplx module}
This module can be summarised very easily: we represent a complex number with a tuple which looks like $\{:cpx, r, i\}$, where $r$ is the real part of the complex number, and $i$ is the imaginary part. The module contains FOUR functions: $new$ (creates a new complex number given value of $r$ and coefficient of $i$), $add$(which adds two complex numbers together), $abs$ which gives us the absolute value of a complex number, and $sqr$ which gives us the square of a complex number. An example of a function in this module is shown below:
\begin{verbatim}
def sqr(a) do
    {:cpx, r, i} = a
    {:cpx, (r*r) + (i*i * -1), (2*r*i)}
end
\end{verbatim}


\section*{Brot module}
This is the module which is responsible for checking whether a given complex number is part of the Mandelbrot set or not. It has two functions: $mandelbrot$ and $test$.
The $mandelbrot$ function, given the complex number $c$ and the maximum number of iterations $m$, returns the value $i$ at which $\lvert z_i \rvert > 2$ or 0 if it doesn't for any  $i < m$.

The $test$ function $test$ functions check if we have reached the maximum iteration, in which case it returns zero, or if the absolute value of $z$ is greater than 2, it returns $i$. It also has one base case, where it returns 0 if $i = m$.
\begin{verbatim}
def test(i, z, c, m) do
        a = Cmplx.abs(z)

        if a <= 2.0 do 
            z1 = Cmplx.add(Cmplx.sqr(z), c)
            test(i+1, z1, c, m)
        else
            i
        end
end
\end{verbatim}

\section*{Colors module}
This module only has one function, \textbf{convert}, which, given a depth from zero to max, gives us a color. 

It does so by calculating a variable $f$ (depth/max), which normalizes the depth to a value in the range {[0, 4]}; the variable $a$ then is used to scale $f$ to a value in the range {[0, 16]}; $x$ then rounds $a$ down to the nearest integer; $y$ is then used to used to calculate the fractional value of $a$ and scale it to the range {[0, 255]}. Finally a tuple of the following form is returned: \{:rgb, 0, 0, 255 - y\}. This tuple can be played around with to change the colors of the image produced.

\section*{Mandel module}
This is the main module of the four mentioned so far; it's this module that is responsible for "calculating" the image. Here we have three functions: $mandelbrot$, $rows$ and $row$.

The $mandelbrot$ function takes parameters $width$, $height$ of the image, the coordinates $x$ and $y$ of the center of the image, the scale $k$, and the $depth$ parameter. It then initializes a transformation function $trans$ that maps pixel coordinates to complex numbers. Then, it starts generating the Mandelbrot set image by processing rows:
\begin{verbatim}
def mandelbrot(width, height, x, y, k, depth) do
        trans = fn(w, h) ->
            Cmplx.new(x + k * (w - 1), y - k * (h - 1))
        end

        rows(width, height, trans, depth, [])
end
\end{verbatim}

The $rows$ function generates rows of the Mandelbrot set image. It takes parameters $width$. $height$ of the image, the transformation function $tr$, the $depth$ parameter, and the accumulated $rows$. It recursively generates each row of the image and accumulates them. It also has e base case which just returns the rows if the height of the image is 0.
\begin{verbatim}
def rows(w, h, tr, depth, rows) do
        row = row(w, h, tr, depth, [])
        rows(w, h - 1, tr, depth, [row | rows])
end
\end{verbatim}

The $row$ function generates a single row of the Mandelbrot set image. It takes parameters current width $w$, height $h$, the transformation function $tr$, the $depth$ parameter, and the accumulated $row$. It calculates the complex number corresponding to the current pixel using the $tr$ function, calculates the Mandelbrot depth for that complex number, converts the depth to a color, and recursively generates the remaining pixels in the row. It also has a base case which just returns the row if the width is 0.
\begin{verbatim}
def row(w, h, tr, depth, row) do 
        c = tr.(w, h)
        res = Brot.mandelbrot(c, depth)
        color = Color.convert(res, depth)
        row(w - 1, h, tr, depth, [color | row])
end
\end{verbatim}


\section*{Final image}
The final image produced, using the tuple \{:rgb, 0, 0, 255 - y\} in the Colors module is shown below:
\begin{figure}
    \centering
    \includegraphics{small (1).jpg}
\end{figure}

\textbf{NOTE:} the PPM nodule will not be explained in detail as the code is given to us; it has the $demo$ function which calls the  $small$ (method responsible for handling the dimensions and generation of the image) and stores it in a file.
















\end{document}
