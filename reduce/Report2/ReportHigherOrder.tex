\documentclass[a4paper,11pt]{article}

\usepackage[utf8]{inputenc}
\usepackage{subcaption}
\usepackage{booktabs}
\usepackage{caption}
\usepackage{amsmath}
\usepackage{tkz-euclide}
\usepackage{hyperref}

\begin{document}

\title{
    \textbf{Recursive Functions}
}
\author{Dhruba Das}
\date{Spring Term 2024}

\maketitle

\section*{Introduction}
This assignment is a continuation of the previous assignment, Operation-on-Lists. In this assignment, we have two problems: Implementing higher order functions $map$, $reduce$ and $filter$, AND re-implementing the ten functions which we implemented in the previous assignment, but using our higher order functions.

\section*{Higher order functions}
In this assignment, we encounter the \textbf{three} different patterns once again, which we use to write our $map$, $reduce$ and filter functions. To re-iterate, the first pattern defines functions which take a list and returns a new list containing values which were taken from the original list and/or modified. This is done so by the function $map$ function, which takes a list, performs a function on that list, and returns a new list containing the resulting values:
\begin{verbatim}
def map([], _func) do [] end
  def map([h|t], func) do
    [func.(h)| map(t, func)]
  end
\end{verbatim}

The second pattern defines functions which take a list, and an integer which is used to modify the values in the original list, and return a new list containing the modified values. This is done by the $filter$ function. The $filter$ function takes a list, and a function which returns true or false; then, the elements in the list which match the given criteria of the function, are returned in a new list: 
\begin{verbatim}
def filter([], _func) do [] end
  def filter([h|t], func) do
    case func.(h) do
      true -> [h|filter(t, func)]
      false -> filter(t, func)
    end
  end
\end{verbatim}

The third pattern defines a function which takes a list and returns an accumulated value obtained from all the values in the list. This is done by the $reduce$ function, which takes a list, an accumulator $acc$, and a function, and then returns an accumulated value obtained by performing the given function on all the values in the list:
\begin{verbatim}
def reducel([], acc, _func) do acc end
  def reducel([h|t], acc, func) do
    reducel(t, func.(h, acc), func)
  end
\end{verbatim} 
The code shown above is $reducel$, which is the tail recursive variant of the $reduce$ function.


\section*{Re-implementation}
I will not be going through all 10 functions, but shown below are example-functions for each pattern:
\begin{verbatim}
def inc(list, val) do
    map(list, fn(x) -> x + val end)
  end
  ...
  def even(list) do
    filter(list, fn(x) -> rem(x, 2) == 0 end)
  end
  ...
  def lengt(list) do
    reducel(list, 0, fn(_, b) -> 1 + b end)
  end
\end{verbatim} 


\section*{Pipe operator}
The instructions as to how to use the pipe operator are given to us, so I'll not be showing the original $square$ function; instead shown below is the function $squarepipe$, which is the same function but is defined using pipe operators instead. The $squarepipe$ function takes a list of integers and returns the sum of the square of all values less than $n$.
\begin{verbatim}
def square_pipe(list, n) do
    list |>
      filter(fn(x) -> x < n end) |>
      map(fn(x) -> x * x end) |>
      reducel(0, fn(x, y) -> x + y end)
  end
\end{verbatim} 

\section*{An example}
Shown below is an a working example of the $squarepipe$ function:
\begin{verbatim}
[High]
iex(9)> list
[1, 2, 3, 4, 5, 6, 7, 8, 9, 10]
...
iex(21)> High.square_pipe(list, 3)
5
\end{verbatim} 




\end{document}
